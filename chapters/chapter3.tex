\chapter{The Core Theorems}

We need to prove that RSA can do encryption and decryption correctly, so some math theorems are required.

\section{Fermat's Little Theorem}

\begin{them}[Fermat's Little Theorem]\cite{rosen2010}
    If $p$ is a prime and $a$ is an integer coprime with it, then
    \begin{equation*}
        a^{p - 1} \equiv 1 \pmod{p}
    \end{equation*}
\end{them}

This theorem shows that there exists a number where exponentiation returns the value to the identity.

However, because the number we handle in RSA is formed like $n = pq$, a more general theorem is needed.

\section{Euler's Theorem}

\begin{them}[Euler's Theorem]\cite{rosen2010}
    For any integer $a$ and positive integer $n$ coprime with it,
    $a^{\varphi(n)} \equiv 1 \pmod{n}$, where $\varphi(n)$ is Euler's Totient Function, defined at \ref{def:0}.
\end{them}

\begin{proof}
    Let $R = \{r_1, r_2, \ldots, r_{\varphi(n)}\}$ be the set of positive integers less than $n$ and coprime to $n$.
    Consider the set $S = \{ar_1, ar_2, \ldots, ar_{\varphi(n)}\}$.

    Since $\gcd(a, n) = 1$, for any $r_i \in R$, we have $\gcd(ar_i, n) = 1$.
    Also, if $ar_i \equiv ar_j \pmod{n}$, then since $a$ is invertible modulo $n$, we have $r_i \equiv r_j \pmod{n}$, which implies $i = j$.
    So, the elements of $S$ are distinct modulo $n$ and are a permutation of $R$ modulo $n$.

    Therefore, the product of elements in $S$ is congruent to the product of elements in $R$ modulo $n$:
    \begin{equation*}
        \prod_{i=1}^{\varphi(n)} (ar_i) \equiv \prod_{i=1}^{\varphi(n)} r_i \pmod{n}
    \end{equation*}
    \begin{equation*}
        a^{\varphi(n)} \prod_{i=1}^{\varphi(n)} r_i \equiv \prod_{i=1}^{\varphi(n)} r_i \pmod{n}
    \end{equation*}

    Let $P = \prod_{i=1}^{\varphi(n)} r_i$. Since each $r_i$ is coprime to $n$, $P$ is also coprime to $n$, so $\gcd(P, n) = 1$ and $P$ is invertible modulo $n$.
    Multiplying both sides by $P^{-1}$, we have:
    \begin{equation*}
        a^{\varphi(n)} \equiv 1 \pmod{n}
    \end{equation*}
\end{proof}

\section{Theoretical Application to the RSA Identity}

The correctness of RSA relies on the fact that decryption reverses encryption. Specifically, given a public key $(e, n)$ and private key $(d, n)$, for any message $m$, we must prove that:
\begin{equation*}
    (m^e)^d \equiv m \pmod{n}
\end{equation*}

This is equivalent to proving $m^{ed} \equiv m \pmod{n}$, knowing that $ed \equiv 1 \pmod{\varphi(n)}$.

\begin{proof}
    Since $ed \equiv 1 \pmod{\varphi(n)}$, there exists an integer $k$ such that $ed = 1 + k\varphi(n)$.
    We verify the congruence $m^{ed} \equiv m \pmod{n}$ in two cases, depending on if $m$ and $n$ are coprime.

    \textbf{Case 1: $\gcd(m, n) = 1$}

    In this case, we can directly apply Euler's Theorem ($m^{\varphi(n)} \equiv 1 \pmod{n}$):
    \begin{align*}
        m^{ed} &= m^{1 + k\varphi(n)} \\
               &= m \cdot (m^{\varphi(n)})^k \\
               &\equiv m \cdot (1)^k \pmod{n} \\
               &\equiv m \pmod{n}
    \end{align*}

    \textbf{Case 2: $\gcd(m, n) \neq 1$}

    Since $n = pq$ is a product of two distinct primes, if $\gcd(m, n) \neq 1$, then $m$ must be a multiple of $p$ or $q$.

    Without loss of generality, assume $m$ is a multiple of $p$, i.e., $m \equiv 0 \pmod{p}$. Then trivially:
    \begin{equation*}
        m^{ed} \equiv 0 \equiv m \pmod{p}
    \end{equation*}

    If $m$ is also a multiple of $q$, then $m$ is a multiple of $pq=n$, so $m^{ed} \equiv 0 \equiv m \pmod{n}$.

    If $m$ is not a multiple of $q$, then $\gcd(m, q) = 1$. By Fermat's Little Theorem ($m^{q-1} \equiv 1 \pmod{q}$):
    \begin{align*}
        m^{ed} &= m^{1 + k\varphi(n)} \\
               &= m^{1 + k(p-1)(q-1)} \\
               &= m \cdot (m^{q-1})^{k(p-1)} \\
               &\equiv m \cdot (1)^{k(p-1)} \pmod{q} \\
               &\equiv m \pmod{q}
    \end{align*}

    Since $m^{ed} \equiv m \pmod{p}$ and $m^{ed} \equiv m \pmod{q}$, and $\gcd(p, q) = 1$, by the Chinese Remainder Theorem properties, we conclude:
    \begin{equation*}
        m^{ed} \equiv m \pmod{pq}
    \end{equation*}
\end{proof}
