\chapter{The RSA Algorithm and Proof of Correctness}

Building upon the properties of the multiplicative group of integers modulo $n$
($\mathbb{Z}_n^*$) and
Euler's theorem established in the previous chapter,
we can now construct the RSA algorithm.
This process relies on the difficulty of reversing operations within this algebraic structure
without specific ``trapdoor'' information.

\section{Key Generation Process}

The security of RSA begins with the careful construction of a finite cyclic group and the selection of exponents.
The process generates a public key (distributed openly) and a private key (kept secret).

\begin{enumerate}
	\item \textbf{Select Primes}

	      Choose two large prime numbers, $p$ and $q$.

	\item \textbf{Compute Modulus}

	      Calculate $n=p\cdot q$,
	      which will be the modulus for both the public and private keys.
	      The size of $n$ in bits determines the security level of the RSA algorithm.

	\item \textbf{Compute Totient}

	      Calculate Euler's totient function
	      \begin{equation}
		      \varphi(n)=\varphi(p\cdot q)=\varphi(p)\cdot \varphi(q)=(p-1)(q-1).
	      \end{equation}
	      This value represents the order of the group $\mathbb{Z}_n^*$.

	\item \textbf{Choose Public Exponent}

	      Select an integer $e\in(1,\varphi(n))$ such that $e\perp\varphi(n)$,
	      i.e. $\gcd(e,\varphi(n))=1$.
	      This ensures that $e$ has a multiplicative inverse modulo $\varphi(n)$.

	\item \textbf{Compute Private Exponent}

	      Apply the Extended Euclidean Algorithm to determine $d$ as the modular multiplicative inverse of $e$ modulo $\varphi(n)$.
	      \begin{equation}
		      d\equiv e^{-1}\pmod{\varphi(n)}.
	      \end{equation}
\end{enumerate}

As a result of the process described above,
the public key is $(e,n)$,
while the private key is $d$.

\section{Encryption and Decryption}

Assume the message $m$ is an integer in the range $[0,n)$.
The sender uses the recipient's public key $(e,n)$ to compute the ciphertext $c$ as follows:
\begin{equation}
	c\equiv m^e\pmod{n}.
\end{equation}

The recipient uses their private key $d$ to recover the original message $m$ from the ciphertext $c$:
\begin{equation}
	m\equiv c^d\pmod{n}.
\end{equation}

\section{Proof of Correctness}
For the RSA cryptosystem to be valid,
we must prove that the encryption-decryption process actually recovers any original message $m$.

\begin{them}[RSA Correctness]
	Given the public key $(e,n)$ and private key $d$ generated as described,
	we must prove that
	\begin{equation}
		\forall m\in\mathbb{N}\cap[0,n)\ ((m^e)^d\equiv m\pmod{n}).
	\end{equation}
\end{them}

\begin{proof}
	By definition,
	$d$ is the modular multiplicative inverse of $e$ modulo $\varphi(n)$,
	i.e.
	\begin{equation}
		d\cdot e\equiv 1\pmod{\varphi(n)}.
	\end{equation}
	This implies that there exists some integer $i$ such that
	\begin{equation}
		d\cdot e=i\cdot\varphi(n)+1=i(p-1)(q-1)+1.
	\end{equation}
	Rewrite the left-hand side of the equation to be proved:
	\begin{equation}
		(m^e)^d=m^{e\cdot d}=m^{i(p-1)(q-1)+1}=m^{i(p-1)(q-1)}\cdot m.
	\end{equation}

	Firstly, let's prove $(m^e)^d\equiv m\pmod{p}$.
	Examine two possible cases for the relation between $p$ and $m$:

	\begin{case}[$p\nmid m$]
		By Fermat's Little Theorem,
		since $p$ is prime and $p\nmid m$,
		\begin{equation}
			m^{p-1}\equiv 1\pmod{p}.
		\end{equation}
		Therefore,
		\begin{equation}
			m^{d\cdot e}=m^{i(p-1)(q-1)}\cdot m=(m^{p-1})^{i(q-1)}\cdot m\equiv 1^{i(q-1)}\cdot m\equiv m\pmod{p}.
		\end{equation}
	\end{case}

	\begin{case}[$p\mid m$]
		In this case,
		$m\equiv 0\pmod{p}$.
		Thus the congruence
		\begin{equation}
			m^{d\cdot e}=m^{i(p-1)(q-1)}\cdot m\equiv 0^{i(p-1)(q-1)}\cdot 0\equiv 0\pmod{p}
		\end{equation}
		also holds.
	\end{case}

	In both cases,
	$m^{d\cdot e}\equiv m\pmod{p}$.

	Due to the symmetry between $p$ and $q$,
	the exact same logic applies to $q$.
	Thus,
	$m^{d\cdot e}\equiv m\pmod{q}$ can be established similarly.

	The derivation above establishes two congruences:
	\begin{equation}
		\begin{cases}
			m^{d\cdot e}\equiv m\pmod{p}\iff m^{d\cdot e}-m\equiv 0\pmod{p}\iff p\mid (m^{d\cdot e}-m) \\
			m^{d\cdot e}\equiv m\pmod{q}\iff m^{d\cdot e}-m\equiv 0\pmod{q}\iff q\mid (m^{d\cdot e}-m)
		\end{cases}.
	\end{equation}

	Since $p$ and $q$ are distinct primes,
	they are coprime, i.e. $\gcd(p,q)=1$.
	By applying the property of divisibility,
	\begin{equation}
		n=p\cdot q\mid(m^{d\cdot e}-m),
	\end{equation}
	which is equivalent to
	\begin{equation}
		(m^e)^d\equiv m\pmod{n}.
	\end{equation}
\end{proof}
