\chapter{The RSA Algorithm and Proof of Correctness}

Building upon the properties of the multiplicative group of integers modulo $n$
($\mathbb{Z}_n^*$) and
Euler's theorem established in the previous chapter,
we can now construct the RSA algorithm.
This process relies on the difficulty of reversing operations within this algebraic structure
without specific ``trapdoor'' information.

\section{Key Generation Process}

The security of RSA begins with the careful construction of a finite cyclic group and the selection of exponents.
The process generates a public key (distributed openly) and a private key (kept secret).

\begin{enumerate}
	\item \textbf{Select Primes}

	      Choose two large prime numbers, $p$ and $q$.

	\item \textbf{Compute Modulus}

	      Calculate $n=p\cdot q$,
	      which will be the modulus for both the public and private keys.
	      The size of $n$ in bits determines the security level of the RSA algorithm.

	\item \textbf{Compute Totient}

	      Calculate Euler's totient function
	      $$
		      \varphi(n)=\varphi(p\cdot q)=\varphi(p)\cdot \varphi(q)=(p-1)(q-1).
	      $$
	      This value represents the order of the group $\mathbb{Z}_n^*$.

	\item \textbf{Choose Public Exponent}

	      Select an integer $e\in(1,\varphi(n))$ such that $e\perp\varphi(n)$,
	      i.e. $\gcd(e,\varphi(n))=1$.
	      This ensures that $e$ has a multiplicative inverse modulo $\varphi(n)$.

	\item \textbf{Compute Private Exponent}

	      Apply the Extended Euclidean Algorithm to determine $d$ as the modular multiplicative inverse of $e$ modulo $\varphi(n)$.
	      $$
		      d\equiv e^{-1}\pmod{\varphi(n)}.
	      $$
\end{enumerate}

As a result of the process described above,
the public key is $(e,n)$,
while the private key is $d$.