\chapter{Mathematical Foundation}

To build the RSA algorithm and verify its correctness, some basic mathematical tools are needed.

\section{Modular Arithmetic and the Multiplicative Group}

\begin{defn}
  If $a$ and $b$ are integers and $m$ is a positive integer, then $a$ is \textbf{congruent} to $b$ modulo $m$ if $m$ divides the difference $a - b$. We use the notation $a \equiv b \pmod{m}$ to indicate it.
\end{defn}

From the definition of congruence, we have Theorem \ref{them:0}.

\begin{them}\label{them:0}
  $a \equiv b \pmod{m} \iff \exists k \in \mathbb{Z} \text{ such that } a = b + km$.
\end{them}

\begin{proof}
  This proof is straightforward from the definition.
  
  $(\Rightarrow)$ If $a \equiv b \pmod{m}$, then $m$ divides the difference $a - b$. So we have an integer $k$ such that $a - b = km$, which implies $a = b + km$.

  $(\Leftarrow)$ If $a = b + km$, then $a - b = km$, which means $m$ divides $a - b$. So $a \equiv b \pmod{m}$.
\end{proof}

Let $\mathbb{Z}_m$ be the set of all possible remainders modulo $m$, as the set $\{n \in \mathbb{Z} \mid 0 \le n < m\}$. We can define the addition and multiplication on this set, as

\begin{align*}
  a +_m b &= (a + b) \pmod{m} \\
  a \cdot_m b &= (a \cdot b) \pmod{m}
\end{align*}

These operations form the arithmetic modulo $m$. What's more, by considering the set $\{n\in \mathbb{Z}_m \ |\ \gcd{} (n, m) = 1\}$, we can form the multiplicative group of integers modulo $m$.

\begin{defn}[Multiplicative group of integers modulo $m$]
  The group $(\mathbb{Z}_m^*, \cdot_m)$ consists of the set $\mathbb{Z}_m^* = \{k \in \mathbb{Z}_m \mid \gcd(k, m) = 1\}$, with the operation $a \cdot_m b = (a \cdot b) \pmod{m}$.
\end{defn}


This group provides a closed, reversible structure, which is essential for making RSA working correctly.

\section{Greatest Common Divisor and the Extended Euclidean Algorithm}

\begin{defn}[Greatest common divisor]
  Let $a$ and $b$ be non-zero integers. The largest integer $d$ such that $d \mid a$ and $d \mid b$ is called the \textbf{greatest common divisor} of $a$ and $b$, denoted by $\gcd(a, b)$.
\end{defn}

\begin{them}[Bézout's Identity]\label{them:1}
  Let $d = \gcd(a, b)$. Then $\exists x, y$ such that $ax + by = d$.
\end{them}

\begin{proof}
  Consider the set $S = \{ax + by \mid x, y \in \mathbb{Z}, ax + by > 0\}$. By the Well-Ordering Principle of integers, there must be a smallest positive element in $S$. Let us name it $d'$.
  
  We first show $d'$ divides $a$. If not, we can write $a = qd' + r$ with $0 < r < d'$. Since $d'$ is in $S$, $d' = ax_0 + by_0$. Then the remainder $r = a - q(ax_0 + by_0) = a(1 - qx_0) + b(-qy_0)$. So $r$ is also a linear combination of $a$ and $b$, and is positive and smaller than $d'$. This contradicts the fact that $d'$ is the smallest. So $d'$ must divide $a$. Similarly, $d'$ divides $b$.
  
  Since $d'$ is a common divisor, and $d$ is the greatest common divisor, we have $d' \le d$.
  
  On the other hand, since $d \mid a$ and $d \mid b$, and $d' = ax_0 + by_0$, we know $d$ divides $d'$, so $d \le d'$.
  
  Therefore $d = d'$, so $d$ can be written as $ax + by$.
\end{proof}

By using the Extended Euclidean Algorithm, we can not only find the greatest common divisor of two integers $a$ and $b$, but also determine the $x$ and $y$ by theorem\ref{them:1}. 

\begin{lstlisting}[caption={The Extended Euclidean Algorithm}, label={alg:eea_general}]
Algorithm: EEA(a, b)
Input: Two integers a and b
Output: A tuple (gcd, x, y) such that ax + by = gcd(a, b)

$old\_r, r \gets a, b$
$old\_s, s \gets 1, 0$
$old\_t, t \gets 0, 1$

While $r \neq 0$:
      $q \gets old\_r \text{ div } r$
      $(old\_r, r) \gets (r, old\_r - q \times r)$
      $(old\_s, s) \gets (s, old\_s - q \times s)$
      $(old\_t, t) \gets (t, old\_t - q \times t)$

Return $(old\_r, old\_s, old\_t)$
\end{lstlisting}

\section{Euler's Totient Function}

\begin{defn}[Euler's Totient Function]\label{def:0}
  Euler's Totient Function $\varphi(n)$ counts the number of integers less than $n$ that are coprime to $n$.
\end{defn}

A useful expression for calculating this function based on the prime factorization of $n$ is given by Theorem \ref{them:2}.

\begin{them}\label{them:2}
  \begin{equation*}
    \varphi(n) = \prod_{i = 1}^r p_i^{k_i - 1}(p_i - 1)
  \end{equation*}
  where $n = p_1^{k_1}p_2^{k_2}\cdots p_r^{k_r}$.
\end{them}

\begin{proof}
  Since Euler's Totient Function is multiplicative, which means if $\gcd(n, m) = 1$, then $\varphi(nm) = \varphi(n)\varphi(m)$, we can calculate $\varphi(n)$ by calculating each prime power factor.

  Consider $n = p^k$ where $p$ is a prime. The integers between $1$ and $p^k$ that are not coprime to $p^k$ must be multiples of $p$.
  These multiples are $1\cdot p, 2\cdot p, \ldots, p^{k-1}\cdot p$. There are exactly $p^{k-1}$ such numbers.
  
  So the count of numbers that are coprime to $p^k$ is the total numbers minus these multiples:
  \begin{equation*}
    \varphi(p^k) = p^k - p^{k-1} = p^{k-1}(p - 1)
  \end{equation*}

  Now for $n = p_1^{k_1}p_2^{k_2}\cdots p_r^{k_r}$, since different prime powers are coprime to each other, we have:
  \begin{align*}
    \varphi(n) &= \varphi(p_1^{k_1})\varphi(p_2^{k_2})\cdots\varphi(p_r^{k_r}) \\
               &= \prod_{i=1}^r p_i^{k_i - 1}(p_i - 1)
  \end{align*}
\end{proof}

From theorem\ref{them:2}, we have two useful corollary as followed,

\begin{cor}
  $\varphi(n) = n - 1$, if n is a prime.
\end{cor}

\begin{proof}
  If $n$ is a prime, the only factors of $n$ are $1$ and $n$. The integers less than $n$ are $1, 2, ..., n-1$. None of them divides $n$.
  So, all $n-1$ integers are coprime to $n$. Therefore, $\varphi(n) = n - 1$.
\end{proof}

\begin{cor}
  $\varphi(n) = (p - 1)(q - 1)$, if $n = pq$ and $p, q$ are primes.
\end{cor}

\begin{proof}
  Since $p$ and $q$ are distinct primes, $\gcd(p, q) = 1$.
  By the multiplicative property of Euler's function:
  \begin{equation*}
    \varphi(pq) = \varphi(p)\varphi(q)
  \end{equation*}
  From the standard property of prime numbers, $\varphi(p) = p-1$ and $\varphi(q) = q-1$.
  Substituting these values, we get:
  \begin{equation*}
    \varphi(n) = (p - 1)(q - 1)
  \end{equation*}
\end{proof}
