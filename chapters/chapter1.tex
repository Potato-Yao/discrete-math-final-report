\chapter{Introduction}

In the history of cryptography,
few innovations have been as transformative as the introduction of public-key encryption.
Before the 1970s,
secure communication relied entirely on symmetric keys,
requiring parties to exchange secrets beforehand ---
a logistical vulnerability in an expanding digital world.
In 1977,
Ron Rivest, Adi Shamir, and Leonard Adleman proposed the RSA algorithm,
a method that fundamentally shifted this paradigm by utilizing a pair of keys:
a public key for encryption and a private key for decryption.
Today,
RSA remains one of the most widely deployed cryptosystems,
securing everything from web traffic (e.g. the HTTPS protocol) to digital signatures.

This essay applies the theoretical frameworks learned in our discrete mathematics course ---
specifically, Algebraic Structures and Number Theory ---
to deconstruct the RSA algorithm.
Our objective is to move beyond the "black box" understanding of encryption and
analyze the specific mathematical machinery that makes RSA both functional and secure.
Through rigorous analysis,
we aim to demonstrate how abstract mathematical concepts translate into concrete solutions
for real-world problems in cybersecurity.