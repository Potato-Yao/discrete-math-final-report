%%
% BIThesis 本科毕业设计论文模板(全英文) —— 使用 XeLaTeX 编译 The BIThesis Template for Undergraduate Thesis
% This file has no copyright assigned and is placed in the Public Domain.
%%

% 摘要若要按经管学院的要求,先英文再中文,请调换以下 abstract、abstractEn 的顺序。

\begin{abstract}
  为深入理解公钥加密系统的数学机理,
  基于离散数学中的代数结构与数论框架,
  对RSA算法进行了系统的理论解构。
  首先,
  利用模运算、乘法群及欧拉函数构建了算法的数学基础;
  其次,通过费马小定理与欧拉定理,
  严谨证明了RSA密钥生成、加密及解密过程的正确性,
  并特别推导了消息与模数不互质这一边界情况下的同余关系。
  此外,
  从时间复杂度与安全性维度进行分析,
  论证了模幂运算作为陷门函数的计算不对称性,
  并评估了算法在面对大整数分解攻击(如GNFS)时的安全性。
  研究表明,
  抽象的群论性质与同余理论是保障现代数字基础设施安全的关键支撑,
  离散数学理论在网络安全领域具有决定性的应用价值。
\end{abstract}

% 如需手动控制换行连字符位置,可写 aa\-bb,详见
% https://bithesis.bitnp.net/faq/hyphen.html

\begin{abstractEn}
  To explore the mathematical principles underlying public-key encryption,
  this paper deconstructs the RSA algorithm based on the frameworks of
  algebraic structures and number theory from discrete mathematics.
  The study first establishes the mathematical foundation of the algorithm using
  modular arithmetic, multiplicative groups, and Euler's totient function.
  By applying Fermat's Little Theorem and Euler's Theorem,
  we rigorously prove the correctness of the key generation, encryption, and decryption processes,
  specifically addressing the edge cases where the message and the modulus are not coprime.
  Furthermore,
  the algorithm is analyzed from the perspectives of time complexity and security,
  demonstrating the computational asymmetry of modular exponentiation as a trapdoor function and
  evaluating its resilience against integer factorization attacks
  such as the General Number Field Sieve (GNFS).
  The results confirm that the abstract properties of group theory and congruences
  provide the essential theoretical support for the security of modern digital infrastructure,
  highlighting the critical application value of discrete mathematics in cybersecurity.
\end{abstractEn}