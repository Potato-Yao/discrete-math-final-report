%%
% BIThesis 本科毕业设计论文模板(全英文) —— 使用 XeLaTeX 编译 The BIThesis Template for Undergraduate Thesis
% This file has no copyright assigned and is placed in the Public Domain.
%%

% 摘要若要按经管学院的要求,先英文再中文,请调换以下 abstract、abstractEn 的顺序。

\begin{abstract}
  RSA 算法是密码学中一个重要且基础的算法,在各领域有相当广泛的应用。
  本文旨在通过离散数学课程知识进行理论推导,给出 RSA 算法的流程,证明其正确性,并研究其复杂度及安全性。

  本文首先进行了裴蜀定理的证明,以此给出当正整数 $x$ 在模 $n$ 剩余系下存在乘法逆元时,
  以 $O((\ln n)^2)$ 时间复杂度求解其逆元值的扩展欧几里得算法。
  然后研究了模意义剩余系的性质以证明欧拉定理,进而给出了 RSA 算法的流程并证明了正确性。
  最后,本文分析出 RSA 算法的加密过程为多项式复杂度,而现有的破解算法均为指数级复杂度,从而说明了算法的安全性。

  本文得出结论:
  RSA 算法基于数论中的欧拉定理,通过素性测试、求解乘法逆元等算法,构建了一套密码体系。
  该算法通过一个单向陷门函数,使得加密和解密过程均可在多项式时间内完成,
  但破解过程暂无多项式复杂度算法,这说明了其在现有工程实现下的安全性。
\end{abstract}

% 如需手动控制换行连字符位置,可写 aa\-bb,详见
% https://bithesis.bitnp.net/faq/hyphen.html

\begin{abstractEn}
  RSA Algorithm is an important and fundamental algorithm in cryptography, which is widely used in various fields.
  In this paper, we derive the RSA Algorithm process based on the knowledge of discrete mathematics, 
  prove the correctness of the algorithm, and analyze its time complexity and security.

  We first prove the correctness of Bézout's Theorem to 
  derive the extended Euclidean algorithm for solving the multiplicative inverse of a certain integer $x$ modulo $n$.
  Then, we prove the correctness of Euler's theorem through 
  the properties of the modular arithmetic and the multiplicative groups, 
  and construct the RSA Algorithm process based on the knowledge of Euler's theorem.
  Finally, we analyze the time complexity of the RSA Algorithm and prove its security.

  We reach the conclusions:
  RSA Algorithm is theoretically based on Euler's Theorem in Number Theory. 
  It constructs a cryptosystem with the help of prime test and multiplicative inverse calcutating algorithms.
  The algorithm is based on a one-way trapdoor function, 
  which means that the encryption process can be done in polynomial time, 
  but the existing deciphering attack requires exponential time.
  Therefore, RSA Algorithm is secure in the existing engineering implementation.
\end{abstractEn}