\begin{conclusion}
In this essay,
we have successfully analyzed the RSA algorithm by
breaking it down into its constituent discrete mathematical components.
By modeling the cryptosystem within the multiplicative group of integers modulo $n$,
we demonstrated that the security of RSA is not arbitrary,
but strictly derived from the properties of Euler's Totient Function and
the difficulty of the Integer Factorization Problem.

Our theoretical analysis confirmed the correctness of the algorithm.
We provided a rigorous proof that
for any public exponent $e$ and private exponent $d$ satisfying $ed \equiv 1 \pmod{\varphi(n)}$
where $n$ is the product of two distinct primes,
the congruence $(m^e)^d \equiv m \pmod n$ holds true.
Crucially,
we showed that this holds even in edge cases where
the message $m$ is not coprime to the modulus $n$,
ensuring the reliability of the system for all possible inputs.

Furthermore,
our investigation into complexity revealed the delicate balance between efficiency and security.
While encryption and decryption operate in polynomial time\\
($O((\ln n)^2)$ and $O((\ln n)^3)$ respectively),
the security relies on the exponential time complexity required to reverse the "trapdoor" function via factorization.
Our review of attack vectors,
including the General Number Field Sieve (GNFS),
suggests that while 2048-bit keys remain secure against current classical computing capabilities,
the potential emergence of quantum computing and Shor's algorithm poses a theoretical existential threat to RSA in the future,
calling for ongoing vigilance, adaptation and innovation in cryptographic practices.

Ultimately,
this study reinforces the vital importance of discrete mathematics in computer science.
The abstract theories of groups, congruences, and prime numbers are not merely academic exercises;
they are the invisible walls that protect our digital infrastructure.
\end{conclusion}